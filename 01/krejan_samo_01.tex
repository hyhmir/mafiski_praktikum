\documentclass[slovene,11pt,a4paper]{article}
\usepackage[margin=2cm,bottom=3cm,foot=1.5cm]{geometry}

\setlength{\parindent}{0pt}
\setlength{\parskip}{0.5ex}

\usepackage[pdftex]{graphicx}
\DeclareGraphicsExtensions{.pdf,.png}


\usepackage{amsmath}
\usepackage{amsfonts}
\usepackage{mathrsfs}
\usepackage[usenames]{color}
\usepackage[slovene]{babel}
\usepackage[utf8]{inputenc}
\usepackage{siunitx}

\def\phi{\varphi}
\def\eps{\varepsilon}
\def\theta{\vartheta}

\newcommand{\thisyear}{2025/26}

\renewcommand{\Re}{\mathop{\rm Re}\nolimits}
\renewcommand{\Im}{\mathop{\rm Im}\nolimits}
\newcommand{\Tr}{\mathop{\rm Tr}\nolimits}
\newcommand{\diag}{\mathop{\rm diag}\nolimits}
\newcommand{\dd}{\,\mathrm{d}}
\newcommand{\ddd}{\mathrm{d}}
\newcommand{\ii}{\mathrm{i}}
\newcommand{\lag}{\mathcal{L}\!}
\newcommand{\ham}{\mathcal{H}\!}
\newcommand{\four}[1]{\mathcal{F}\!\left(#1\right)}
\newcommand{\bigO}[1]{\mathcal{O}\!\left(#1\right)}
\newcommand{\sh}{\mathop{\rm sinh}\nolimits}
\newcommand{\ch}{\mathop{\rm cosh}\nolimits}
\renewcommand{\th}{\mathop{\rm tanh}\nolimits}
\newcommand{\erf}{\mathop{\rm erf}\nolimits}
\newcommand{\erfc}{\mathop{\rm erfc}\nolimits}
\newcommand{\sinc}{\mathop{\rm sinc}\nolimits}
\newcommand{\rect}{\mathop{\rm rect}\nolimits}
\newcommand{\ee}[1]{\cdot 10^{#1}}
\newcommand{\inv}[1]{\left(#1\right)^{-1}}
\newcommand{\invf}[1]{\frac{1}{#1}}
\newcommand{\sqr}[1]{\left(#1\right)^2}
\newcommand{\half}{\frac{1}{2}}
\newcommand{\thalf}{\tfrac{1}{2}}
\newcommand{\pd}{\partial}
\newcommand{\Dd}[3][{}]{\frac{\ddd^{#1} #2}{\ddd #3^{#1}}}
\newcommand{\Pd}[3][{}]{\frac{\pd^{#1} #2}{\pd #3^{#1}}}
\newcommand{\avg}[1]{\left\langle#1\right\rangle}
\newcommand{\norm}[1]{\left\Vert #1 \right\Vert}
\newcommand{\braket}[2]{\left\langle #1 \vert#2 \right\rangle}
\newcommand{\obraket}[3]{\left\langle #1 \vert #2 \vert #3 \right \rangle}
\newcommand{\hex}[1]{\texttt{0x#1}}

\renewcommand{\iint}{\mathop{\int\mkern-13mu\int}}
\renewcommand{\iiint}{\mathop{\int\mkern-13mu\int\mkern-13mu\int}}
\newcommand{\oiint}{\mathop{{\int\mkern-15mu\int}\mkern-21mu\raisebox{0.3ex}{$\bigcirc$}}}

\newcommand{\wunderbrace}[2]{\vphantom{#1}\smash{\underbrace{#1}_{#2}}}

\renewcommand{\vec}[1]{\overset{\smash{\hbox{\raise -0.42ex\hbox{$\scriptscriptstyle\rightharpoonup$}}}}{#1}}
\newcommand{\bec}[1]{\mathbf{#1}}



\title{
\sc\large Matematično-fizikalni praktikum \thisyear\\
\bigskip
\bf\Large 1.~naloga: Airyjevi funkciji
}
\author{Samo Krejan, 28231092}
\date{}

\newcommand{\Ai}{\mathrm{Ai}}
\newcommand{\Bi}{\mathrm{Bi}}
\newcommand{\bi}[1]{\hbox{\boldmath{$#1$}}}
\newcommand{\bm}[1]{\hbox{\underline{$#1$}}}


\begin{document}
\maketitle
\vspace{-1cm}

\section{Uvod}

Airyjevi funkciji $\Ai$ in $\Bi$ % #TODO: add figure
se v fiziki pojavljata predvsem v optiki in kvantni mehaniki
\cite{Airy_use}.  Definirani sta kot neodvisni rešitvi enačbe
%
\begin{equation*}
  y''(x) -xy(x) = 0
\end{equation*}
%
in sta predstavljivi v integralski obliki
%
\begin{equation*}
  \Ai(x) = \frac{1}{\pi} \int_0^\infty \cos (t^3/3 + x t) \dd t \>,\quad
  \Bi(x) = \frac{1}{\pi} \int_0^\infty \left[ \mathrm{e}^{-t^3/3 + x t}
  + \sin (t^3/3 + x t) \right] \dd t \>.
\end{equation*}
%


Za majhne $x$ lahko funkciji $\Ai$ in $\Bi$ izrazimo
z Maclaurinovima vrstama
%
\begin{equation*}
  \Ai(x) = \alpha f(x) - \beta g(x)\>,\qquad
  \Bi(x) = \sqrt{3}\, \Bigl[\alpha f (x) + \beta g(x) \Bigr]\>,
\end{equation*}
kjer v $x=0$ veljata zvezi
%
$\alpha = \Ai(0) = \Bi(0)/\sqrt{3}\approx 0.355028053887817239$ in
$\beta = -\Ai'(0) = \Bi'(0)/\sqrt{3}\approx 0.258819403792806798$.
Vrsti za $f$ in $g$ sta
\begin{equation*}
  f(x) = \sum_{k=0}^\infty
  \left(\frac{1}{3}\right)_k \frac{3^k x^{3k}}{(3k)!} \>, \qquad
  g(x) = \sum_{k=0}^\infty
  \left(\frac{2}{3}\right)_k \frac{3^k x^{3k+1}}{(3k+1)!} \>,
\end{equation*}
kjer je
\begin{equation*}
  (z)_n = \Gamma(z+n)/\Gamma(z) \>, \qquad (z)_0 = 1 \>.
\end{equation*}

Za velike vrednosti $|x|$ Airyjevi funkciji aproksimiramo
z njunima asimp\-tot\-ski\-ma razvojema.  Z novo spremenljivko
$\xi=\frac{2}{3} |x|^{3/2}$ in asimptotskimi vrstami
%
\begin{equation*}
  L(z) \sim \sum_{s=0}^\infty \frac{u_s}{z^s}\>,\qquad
  P(z) \sim \sum_{s=0}^\infty (-1)^s \frac{u_{2s}}{z^{2 s}}\>,\qquad
  Q(z) \sim \sum_{s=0}^\infty (-1)^s \frac{u_{2s+1}}{z^{2 s+1}}\>,
\end{equation*}
s koeficienti
\begin{equation*}
u_s = \frac{ \Gamma(3s + \frac{1}{2})}
        {54^s s!\, \Gamma(s + \frac{1}{2}) }
\end{equation*}
za velike pozitivne $x$ izrazimo
%
\begin{equation*}
\Ai(x)\sim  \frac{\mathrm{e}^{-\xi}}{2\sqrt{\pi} x^{1/4}} \, L(-\xi) \>, \qquad
\Bi(x)\sim  \frac{\mathrm{e}^{\xi}} { \sqrt{\pi} x^{1/4}} \, L(\xi)\>,
\end{equation*}
%
za po absolutni vrednosti velike negativne $x$ pa
%
%
\begin{align*}
    \Ai(x)&\sim  \frac{1}{\sqrt{\pi} (-x)^{1/4}}
    \Bigl[ \phantom{-}\sin(\xi-\pi/4) \, Q(\xi)
                    + \cos(\xi-\pi/4) \, P(\xi)\Bigr] \>, \\
    \Bi(x)&\sim  \frac{1}{\sqrt{\pi} (-x)^{1/4}}
    \Bigl[ - \sin(\xi-\pi/4) \, P(\xi)
      + \cos(\xi-\pi/4) \, Q(\xi)\Bigr]\>.
\end{align*}


\section{Naloga}

Z uporabo kombinacije Maclaurinove vrste in asimptotskega
razvoja poišči čim učinkovitejši postopek za izračun
vrednosti Airyjevih funkcij $\Ai$ in $\Bi$ na vsej real\-ni osi
z {\bf absolutno} napako, manjšo od $10^{-10}$. Enako naredi tudi z {\bf relativno} napako in ugotovi,
ali je tudi pri le-tej dosegljiva natančnost, manjša od $10^{-10}$.
Pri oceni napak si po\-ma\-gaj s programi, ki znajo računati s poljubno
natančnostjo, na primer z {\sc Mathematico} in/ali paketi {\sc mpmath} in {\sc decimal} v programskem
jeziku {\sc Python}.





\section{Dodatna naloga}

Ničle funkcije $\Ai$ pogosto srečamo v matematični
analizi pri določitvi intervalov ničel specialnih funkcij
in ortogonalnih polinomov \cite{1_szego} ter v fiziki pri računu
energijskih spektrov kvantnomehanskih sistemov \cite{1_landauQM}.
Poišči prvih sto ničel $\{a_s\}_{s=1}^{100}$ Airyjeve
funkcije $\Ai$ in prvih sto ničel $\{b_s\}_{s=1}^{100}$
funkcije $\Bi$ pri $x<0$ ter dobljene vrednosti primerjaj s formulama
%
\begin{equation*}
  a_s = - f \left( \frac{3\pi(4s-1)}{8} \right) \>, \qquad
  b_s = - f \left( \frac{3\pi(4s-3)}{8} \right) \>, \qquad s = 1,2,\ldots \>,
\end{equation*}
%
kjer ima funkcija $f$ asimptotski razvoj \cite{1_abram}
%
\begin{equation*}
  f(z) \sim z^{2/3} \left(
  1 + \frac{5}{48} \, z^{-2}
  -\frac{5}{36} \, z^{-4}
  +\frac{77125}{82944} \, z^{-6}
  -\frac{108056875}{6967296} \, z^{-8} + \ldots\right) \>.
\end{equation*}


\bibliographystyle{plain}
\bibliography{references}


\end{document}