\documentclass[slovene,11pt,a4paper]{article}
\usepackage[margin=2cm,bottom=3cm,foot=1.5cm]{geometry}
\setlength{\parindent}{0pt}
\setlength{\parskip}{0.5ex}

\usepackage[pdftex]{graphicx}
\DeclareGraphicsExtensions{.pdf,.png}


\usepackage{amsmath}
\usepackage{amsfonts}
\usepackage{mathrsfs}
\usepackage[usenames]{color}
\usepackage[slovene]{babel}
\usepackage[utf8]{inputenc}
\usepackage{siunitx}

\def\phi{\varphi}
\def\eps{\varepsilon}
\def\theta{\vartheta}

\newcommand{\thisyear}{2025/26}

\renewcommand{\Re}{\mathop{\rm Re}\nolimits}
\renewcommand{\Im}{\mathop{\rm Im}\nolimits}
\newcommand{\Tr}{\mathop{\rm Tr}\nolimits}
\newcommand{\diag}{\mathop{\rm diag}\nolimits}
\newcommand{\dd}{\,\mathrm{d}}
\newcommand{\ddd}{\mathrm{d}}
\newcommand{\ii}{\mathrm{i}}
\newcommand{\lag}{\mathcal{L}\!}
\newcommand{\ham}{\mathcal{H}\!}
\newcommand{\four}[1]{\mathcal{F}\!\left(#1\right)}
\newcommand{\bigO}[1]{\mathcal{O}\!\left(#1\right)}
\newcommand{\sh}{\mathop{\rm sinh}\nolimits}
\newcommand{\ch}{\mathop{\rm cosh}\nolimits}
\renewcommand{\th}{\mathop{\rm tanh}\nolimits}
\newcommand{\erf}{\mathop{\rm erf}\nolimits}
\newcommand{\erfc}{\mathop{\rm erfc}\nolimits}
\newcommand{\sinc}{\mathop{\rm sinc}\nolimits}
\newcommand{\rect}{\mathop{\rm rect}\nolimits}
\newcommand{\ee}[1]{\cdot 10^{#1}}
\newcommand{\inv}[1]{\left(#1\right)^{-1}}
\newcommand{\invf}[1]{\frac{1}{#1}}
\newcommand{\sqr}[1]{\left(#1\right)^2}
\newcommand{\half}{\frac{1}{2}}
\newcommand{\thalf}{\tfrac{1}{2}}
\newcommand{\pd}{\partial}
\newcommand{\Dd}[3][{}]{\frac{\ddd^{#1} #2}{\ddd #3^{#1}}}
\newcommand{\Pd}[3][{}]{\frac{\pd^{#1} #2}{\pd #3^{#1}}}
\newcommand{\avg}[1]{\left\langle#1\right\rangle}
\newcommand{\norm}[1]{\left\Vert #1 \right\Vert}
\newcommand{\braket}[2]{\left\langle #1 \vert#2 \right\rangle}
\newcommand{\obraket}[3]{\left\langle #1 \vert #2 \vert #3 \right \rangle}
\newcommand{\hex}[1]{\texttt{0x#1}}

\renewcommand{\iint}{\mathop{\int\mkern-13mu\int}}
\renewcommand{\iiint}{\mathop{\int\mkern-13mu\int\mkern-13mu\int}}
\newcommand{\oiint}{\mathop{{\int\mkern-15mu\int}\mkern-21mu\raisebox{0.3ex}{$\bigcirc$}}}

\newcommand{\wunderbrace}[2]{\vphantom{#1}\smash{\underbrace{#1}_{#2}}}

\renewcommand{\vec}[1]{\overset{\smash{\hbox{\raise -0.42ex\hbox{$\scriptscriptstyle\rightharpoonup$}}}}{#1}}
\newcommand{\bec}[1]{\mathbf{#1}}


\title{
\sc\large Matematično-fizikalni praktikum \thisyear\\
\bigskip
\bf\Large 2.~naloga: Naključni sprehodi
}
\author{}
\date{}

\newcommand{\bi}[1]{\hbox{\boldmath{$#1$}}}
\newcommand{\bm}[1]{\hbox{\underline{$#1$}}}

\begin{document}
\maketitle
\vspace{-1cm}

Naključni sprehodi so vrsta gibanja, pri katerem
v velikem številu korakov napredujemo iz izhodišča
v neko končno lego, tako da se parametri vsakega naslednjega
koraka sproti naključno določajo.  Običajni zgled
je Brownovo gibanje (difuzija) drobnih delcev barvila
po mirujoči homogeni tekočini, kjer je spočetka barvilo
zbrano v izhodišču.  ``Težišče'' barvila
$\langle x(t)\rangle$ v povprečju ostane v izhodišču,
razen če v tekočini vzpostavimo kako anizotropijo (na primer
v dveh razsežnostih z vsiljeno rotacijo).  ``Razmazanost''
po dolgem času je sorazmerna s časom,
\begin{equation*}
  \sigma^2(t) \equiv \langle x^2(t)\rangle - \langle x(t)\rangle^2 = 2 D t \>.
\end{equation*}
Sorazmernostni koeficient je običajna difuzijska konstanta,
priča smo normalni difuziji.  Ta rezultat izhaja iz
centralnega limitnega teorema (CLT), ki izraža,
da je rezultantna porazdelitev končnih leg pri difuziji
porazdeljena normalno (Gauss), če so le povprečni časi
med koraki in povprečni kvadrati dolžin korakov končni.

Zanimiveje je opazovati naključne sprehode, pri katerih dovolimo
nadpovprečno dolge korake.  Verjetnostno gostoto porazdelitve
po dolžinah posameznih korakov parametrizirajmo v potenčni obliki
\begin{equation}
p(l) \propto l^{-\mu} \>,
\label{lpow}
\end{equation}
kjer naj bo $1 < \mu < 3$.  Tedaj postane drugi moment porazdelitve
\begin{equation*}
  \langle l^2\rangle = \int l^2 p(l) \dd l
\end{equation*}
neskončen.  Govorimo o anomalni difuziji, prisotni pri celi dru\v zini
kinematičnih distribucij dolžin poti z "debelimi repi".

Ustrezno sliko naključnega gibanja, povezanega s temi dolgimi koraki, lahko
interpretiramo na dva načina:
\begin{itemize}
  \item L\'evyjev pobeg oz. polet ({\sl flight\/}), implicira, da vsak korak iz
  porazdelitve~(\ref{lpow}) traja enako dolgo, medtem ko se hitrost gibanja med koraki (divje) spreminja.
  \item L\'evyjev sprehod ({\sl walk\/}), ki interpretira korak iz porazdelitve~(\ref{lpow}) kot  gibanje s konstantno hitrostjo in
  tako koraki trajajo različno dolgo časa (dolžina koraka je sorazmerna s časom).
\end{itemize}

Slednja intepretacija bolj ustreza fizikalni sliki naključnega gibanja delca skozi snov, medtem ko
se prva interpretacija uporablja v druga\v cnih aplikacijah.

Vse naloge lahko obravnavaš za obe interpretaciji, pobegov in sprehodov. V prvem primeru (pobeg, flight) je
prete\v ceni \v cas direktno sorazmeren s \v stevilom korakov, v drugem primeru (sprehod, walk) pa je
prete\v ceni \v cas  sorazmeren z vsoto dol\v zine korakov.


Pri anomalni difuziji razmazanost (varianca) velike množice
končnih leg naključnih L\'evyjevih \textbf{sprehodov (walks)} narašča z drugačno potenco časa.
Velja $\sigma^2(t) \sim t^\gamma$, kjer je
\begin{align*}
1 < \mu < 2 \>, &\qquad \gamma = 2 \> &\qquad&  \text{(balistični režim)}\>, \\
2 < \mu < 3 \>, &\qquad \gamma = 4 - \mu &\qquad&  \text{(super-difuzivni režim)}\>, \\
    \mu > 3 \>, &\qquad \gamma = 1 &\qquad&  \text{(normalna difuzija)} \>.
\end{align*}
Za $\mu=2$ pričakujemo $\sigma^2(t) \sim t^2 / \ln t$,
za $\mu=3$ pa $\sigma^2(t) \sim t \ln t$ (glej na primer \cite{weeks}
in druge reference prav tam).

Slika je nekoliko drugačna pri opazovanju naključnih L\'evyjevih \textbf{poletov (flights)}.
Spet vzamemo zvezo $\sigma^2(t) \sim t^\gamma$ in dobimo odvisnosti
\begin{align*}
1 < \mu < 3 \>, &\qquad \gamma = \frac{2}{\mu-1} \> &\qquad&  \text{(super-difuzivni režim)}\>, \\
    \mu > 3 \>, &\qquad \gamma = 1 &\qquad&  \text{(normalna difuzija)} \>.
\end{align*}
Pri $\mu=2$ očitno pričakujemo $\sigma^2(t) \sim t^2 $, torej balistični režim.
\newline


{\sl Statistični komentar:} v primerih, ko je drugi
moment porazdelitve neskončen, bo tudi račun razmazanosti
končnih leg $x_n$ v obliki
\begin{equation}
\sigma^2 = \frac{1}{N-1} \sum_{n=1}^N \left( x_n - \langle x \rangle \right)^2
\label{sig2}
\end{equation}
divergiral oziroma bo imel ob ponovnih zagonih naključnega sprehoda
močno raztresene vrednosti.  Pomagaš si lahko na več načinov.
širino porazdelitve končnih leg lahko oceniš tako, da prilagajaš
Gaussovo krivuljo zgolj centralnega dela porazdelitve, tako da
s prilagajanjem ne zajameš štrlečih (ne-Gaussovskih) repov.
Lahko tudi neposredno računaš vsoto~(\ref{sig2}), a vanjo
vključiš samo ``razumne'' člene (izpusti na primer nekaj
odstotkov najmanjših in nekaj odstotkov največjih).
Tretja možnost je, da definiramo novo vrsto variance
\begin{equation*}
  \sigma / N^p
\end{equation*}
in poiščemo tako potenco $p$, da ta spremenljivka konvergira
za velike $N$ (oz. velike $t$).  še ena možnost je, da vzameš
kako robustno mero za množico vrednosti $X_i$, na primer MAD,
``median absolute deviation''
\begin{equation*}
  \mathrm{MAD} \equiv \mathrm{median}_i\left( | X_i - \mathrm{median}_j X_j | \right) \>.
\end{equation*}
Z njo merimo povprečje absolutne vrednosti deviacije na način,
ki je zelo malo občutljiv na oddaljene vrednosti v repih porazdelitve,
saj te vrednosti na račun mediane bistveno manj vplivajo kot na
račun običajne povprečne vrednosti.


\bigskip

{\sl Naloga:} Napravi računalniško simulacijo
dvorazsežne naključne hoje za \textbf{polete in sprehode}.  Začni vedno v izhodišču
($x=y=0$), nato pa določi naslednjo lego tako, da naključno
izbereš smer koraka in statistično neodvisno od te izbire
še njegovo dolžino, torej
\begin{eqnarray*}
x &\leftarrow& x + l \cos\phi \>, \\
y &\leftarrow& y + l \sin\phi \>,
\end{eqnarray*}
kjer je $\phi$ enakomerno naključno porazdeljen po intervalu
$[0,2\pi]$, dolžina koraka $l$ pa naj bo porazdeljena
v skladu s potenčno obliko~(Enačba \ref{lpow}). 
Dolžine $l_i$ je v tem primeru potrebno generirati po verjetnostni
porazdelitvi w(l)$\sim$ p(l) (Enačba \ref{lpow}). Za izra\v cun algoritma je osnova naslednja
formula:
\begin{equation}
\int\limits_{a}^{l} w(t)\,  dt = \rho \cdot \int\limits_{a}^{b}  w(t)\, dt,
\end{equation}
ki jo je potrebno re\v siti in iz nje izraziti spremenljivko $l$. Tu je 
$\rho$ (psevdo-)naključno število na intervalu $[0,1]$ ter je $[a,b]$ 
relevantni interval vzorčenja.
Za nekatere porazdelitve je izra\v cun preprost, npr $w(t)=\frac{1}{\tau}
e^{-\frac{t}{\tau}}$ nam da kar:
\begin{equation}
l=-\tau \ln(1-\rho).
\end{equation}


Dodatno pomoč za pretvorbo
med verjetnostnimi porazdelitvami najdeš v gradivu v spletni učilnici 
ter na spletu. 

{\sl Opomba:} Korakaš lahko tudi v kartezičnem sistemu,
\begin{eqnarray*}
x &\leftarrow& x + f_x(\rho_1) \>, \\
y &\leftarrow& y + f_y(\rho_2) \>,
\end{eqnarray*}
kjer sta $\rho_1$ in $\rho_2$ naključni števili na intervalu $[0,1]$
ali $[-1/2,1/2]$, funkciji $f_{x,y}$ pa morata na koncu podati ustrezno 
porazdelitev po dolžinah poti glede na potenčno obliko~(\ref{lpow}), kjer 
lahko določiš zvezo med porazdelitvami po končnih legah $x$ ali $y$ ter  
po $l=\sqrt{x^2+y^2}$ z uporabo ustreznega Jacobijevega faktorja.

V vsakem primeru nariši nekaj značilnih
slik sprehodov za $10$, $100$, $1000$ in $10000$ korakov.
Iz velikega števila sprehodov z velikim številom korakov
nato poskusi določiti eksponent $\gamma$ za nekaj izbranih
parametrov $\mu$ oziroma funkcij $f(x)$ v posameznih primerih
ter presodi, za kakšno vrsto difuzije gre.

\bigskip
{\sl Dodatna naloga:} Naključno spreminjaj še čas,
ko delec pred naslednjim korakom miruje (s tako dodatno
prostostno stopnjo poskušamo modelirati tako imenovani
``sticking time'' ali ``trapping time'' pri anomalni difuziji
elektronov v amorfnih snoveh).  Ustrezna verjetnostna gostota
naj ima potenčno odvisnost
$$
p(t) \propto t^{-\nu} \>,
$$
kjer $1 < \nu < 2$.  Je ta odvisnost razklopljena od porazdelitve
osnovnega naključnega sprehoda po dolžinah (oziroma časih)
posameznih korakov?


\begin{thebibliography}{99}
\setlength{\itemsep}{.2\itemsep}\setlength{\parsep}{.5\parsep}
\bibitem{weeks} E.~R.~Weeks, J.~S.~Urbach, H.~L.~Swinney, Physica D {\bf 97} (1996) 291.
\end{thebibliography}


\end{document}
