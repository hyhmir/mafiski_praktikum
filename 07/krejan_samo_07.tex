\documentclass[slovene,11pt,a4paper]{article}
\usepackage[margin=2cm,bottom=3cm,foot=1.5cm]{geometry}
\setlength{\parindent}{0pt}
\setlength{\parskip}{0.5ex}

\usepackage[pdftex]{graphicx}
\DeclareGraphicsExtensions{.pdf,.png}


\usepackage{amsmath}
\usepackage{amsfonts}
\usepackage{mathrsfs}
\usepackage[usenames]{color}
\usepackage[slovene]{babel}
\usepackage[utf8]{inputenc}
\usepackage{siunitx}
\usepackage{hyperref}
\usepackage{float}

% \def\phi{\varphi}
\def\eps{\varepsilon}
\def\theta{\vartheta}

\newcommand{\thisyear}{2025/26}

\renewcommand{\Re}{\mathop{\rm Re}\nolimits}
\renewcommand{\Im}{\mathop{\rm Im}\nolimits}
\newcommand{\Tr}{\mathop{\rm Tr}\nolimits}
\newcommand{\diag}{\mathop{\rm diag}\nolimits}
\newcommand{\dd}{\,\mathrm{d}}
\newcommand{\ddd}{\mathrm{d}}
\newcommand{\ii}{\mathrm{i}}
\newcommand{\lag}{\mathcal{L}\!}
\newcommand{\ham}{\mathcal{H}\!}
\newcommand{\four}[1]{\mathcal{F}\!\left(#1\right)}
\newcommand{\bigO}[1]{\mathcal{O}\!\left(#1\right)}
\newcommand{\sh}{\mathop{\rm sinh}\nolimits}
\newcommand{\ch}{\mathop{\rm cosh}\nolimits}
\renewcommand{\th}{\mathop{\rm tanh}\nolimits}
\newcommand{\erf}{\mathop{\rm erf}\nolimits}
\newcommand{\erfc}{\mathop{\rm erfc}\nolimits}
\newcommand{\sinc}{\mathop{\rm sinc}\nolimits}
\newcommand{\rect}{\mathop{\rm rect}\nolimits}
\newcommand{\ee}[1]{\cdot 10^{#1}}
\newcommand{\inv}[1]{\left(#1\right)^{-1}}
\newcommand{\invf}[1]{\frac{1}{#1}}
\newcommand{\sqr}[1]{\left(#1\right)^2}
\newcommand{\half}{\frac{1}{2}}
\newcommand{\thalf}{\tfrac{1}{2}}
\newcommand{\pd}{\partial}
\newcommand{\Dd}[3][{}]{\frac{\ddd^{#1} #2}{\ddd #3^{#1}}}
\newcommand{\Pd}[3][{}]{\frac{\pd^{#1} #2}{\pd #3^{#1}}}
\newcommand{\avg}[1]{\left\langle#1\right\rangle}
\newcommand{\norm}[1]{\left\Vert #1 \right\Vert}
\newcommand{\braket}[2]{\left\langle #1 \vert#2 \right\rangle}
\newcommand{\obraket}[3]{\left\langle #1 \vert #2 \vert #3 \right \rangle}
\newcommand{\hex}[1]{\texttt{0x#1}}

\renewcommand{\iint}{\mathop{\int\mkern-13mu\int}}
\renewcommand{\iiint}{\mathop{\int\mkern-13mu\int\mkern-13mu\int}}
\newcommand{\oiint}{\mathop{{\int\mkern-15mu\int}\mkern-21mu\raisebox{0.3ex}{$\bigcirc$}}}

\newcommand{\wunderbrace}[2]{\vphantom{#1}\smash{\underbrace{#1}_{#2}}}

\renewcommand{\vec}[1]{\overset{\smash{\hbox{\raise -0.42ex\hbox{$\scriptscriptstyle\rightharpoonup$}}}}{#1}}
\newcommand{\bec}[1]{\mathbf{#1}}

\title{
\sc\large Matematično-fizikalni praktikum \thisyear\\
\bigskip
\bf\Large 7.~naloga: Newtonov zakon
}
\author{Samo Krejan, 28231092}
\date{}

\begin{document}
\maketitle


Gibanje masne točke v polju sil v eni dimenziji opišemo
z diferencialno enačbo drugega reda, z Newtonovim zakonom
\begin{equation*}
m\, \Dd[2]{x}{t} = F \>.
\end{equation*}
Enačba je seveda enakovredna sistemu enačb prvega reda
\[
m\, \Dd{x}{t} = p \;, \qquad \Dd{p}{t} = F
\]
in tako jo tudi rešujemo: kot sistem dveh enačb prvega reda.

Seveda morajo biti na voljo tudi ustrezni začetni pogoji, tipično
$x(t=0)=x_0$ in $dx/dt=v(t=0)=v_0$. Splošnejše gre tu za sistem diferencialnih
enačb drugega reda:
\[
\Dd[n]{y}{x} = f(x,y,y',y'',...),
\]
ki ga lahko prevedemo na sistem enačb prvega reda z uvedbo novih spremenljivk v slogu
gibalne količine pri Netwonovi enačbi ($y'=v,y''=z,...$).

Z nekaj truda se da eksplicitno dokazati, mi pa lahko privzamemo, da so metode za
reševanje enačb hoda (Runge-Kutta 4. reda, prediktor-korektor... ) neposredno uporabne
za reševanje takšnih sistemov enačb in torej aplikabilne v poljubno dimenzijah, kar
naj bi v principu zadovoljilo večino naših zahtev.

Obstaja še posebna kategorija tako imenovanih \emph{simplektičnih} metod, za enačbe, kjer je $f$ le funkcija koordinat, $f(y)$, ki (približno) ohranjajo tudi Hamiltonian,
torej energijo sistema. Najbolj znana metoda je Verlet/St\"ormer/Encke metoda, ki je globalno
natančna do drugega reda in ki točno ohranja tudi vrtilno količino sistema (če je ta v danem problemu smiselna). Rešujemo torej za vsak diskretni korak $n$ velikosti $h$, $x_n=x_0+n \cdot h$:
\[
\Dd[2]{y}{x} = f(y)
\]
in pri diskretizaciji dobimo recept za korak $y_n$ in $v_n=y'_n$:
\begin{eqnarray*}
y_{n+1} &=& y_n + h \cdot v_n + \frac{h^2}{2} \cdot f(y_n) \\
v_{n+1} &=& v_n +  \frac{h}{2} \cdot \left[ f(y_n) + f(y_{n+1}) \right].
\end{eqnarray*}
Alternativno lahko to shemo zapišemo tudi s pomočjo dodatnih vmesnih točk in preskakujemo med lego in hitrostjo z zamikom h/2 (od tod angleško ime 'leapfrog' za ta zapis):
\begin{eqnarray*}
y_{n+1} &=& y_n + h \cdot v_{n+1/2} \\
v_{n+3/2} &=& v_{n+1/2} + h \cdot f(y_{n+1}).
\end{eqnarray*}


V še enem drugačnem zapisu je metoda poznana tudi kot metoda ``Središčne razlike'' (Central Difference Method, CDM), če nas hitrost ne zanima:
\[
y_{n+1} - 2 y_n + y_{n-1} = h^2 \cdot f(y_n),
\]
kjer prvo točko $y_1$ izračunamo po originalni shemi. Metodo CDM lahko uporabljamo tudi
za primere, ko je f tudi funkcija 'časa' x, f(x,y), le da tu simplektičnost ni zagotovljena
(in tudi verjetno ne relevantna).
Za simplektične metode višjih redov je na voljo na primer Forest-Ruth metoda ali Position
Extended Forest-Ruth Like (PEFRL) metoda, ki sta obe globalno četrtega reda in enostavni za
implementacijo.

\section{Naloga}

Čim več metod uporabi za izračun
nihanja matematičnega nihala z začetnim pogojem  $\theta(0)= \theta_0 = 1$,$\dot{\theta}(0)=0$. 
Poišči korak, ki zadošča za natančnost na 3 mesta. Primerjaj
tudi periodično stabilnost shem: pusti, naj teče račun čez 10
ali 20 nihajev in poglej, kako se amplitude nihajev sistematično
kvarijo. Pomagaš si lahko tudi tako, da občasno izračunaš
energijo $E \propto  1-\cos \theta + \frac{\dot{\theta}^2 }{2 \omega_0^2} $. Nariši tudi
ustrezne fazne portrete!.
Z analitično rešitvijo dobimo za nihajni čas $\frac{4}{\omega_0} K\left(\sin^2\frac{\theta_0}{2}\right)$, kjer je $K(m)$ popolni
eliptični integral prve vrste, ki je v SciPy knjižnici in v članku na spletni učilnici podan z:
\[
K(m)=\int\limits_{0}^{1} \frac{d z}{\sqrt{\left(1-z^{2}\right)\left(1-m z^{2}\right)}} = \int\limits_{0}^{\frac{\pi}{2}} \frac{d u}{\sqrt{\left(1-m \sin^2{u}\right)}}
\] 

\subsection{Rešitve DE}

Za začetek sem si ogledal, kako sploh izgledajo analitične rešitve danega problema. Pri temu sem si, kot omenjeno, pomagal z vgrajeno SciPy funkcijo popolnega eliptičnega intehgrala. Na ta način sem lahko za različne začetne kote (in začetno hitrost enako 0) dobil fazni portret in graf analitične rešitve, prikazana na sliki \ref{anal}

\begin{figure}[h]
    \centering
    \begin{minipage}{0.49\textwidth}
        \centering
        \includegraphics[width=\textwidth]{graphs/analytic.pdf}
        % \caption{Rešitve problema za različne vrednosti parametra $k$ pri začetni temperaturi $21 ^\circ C$}
        % \label{druzina}
    \end{minipage}
    \hfill
    \begin{minipage}{0.49\textwidth}
        \centering
        \includegraphics[width=\textwidth]{graphs/analytic_phase.pdf}
        % \caption{Rešitve problema za različne vrednosti parametra $k$ pri začetni temperaturi $-15 ^\circ C$}
        % \label{druzina 2}
    \end{minipage}
    \label{anal}
    \caption{Analitična rešitev (eliptični integral) in fazni portret (Verlet metoda)}
\end{figure}

Na slikah  lahko opazimo dve zanimivi - povezani stvari, na levi vidimo, da se nihajni čas z večanjem začetnega kota veča, in da parametrična krivulja v faznem prostoru izgleda manj in manj kot krožnica in vedno bolj podobna obliki očesa, dokler se le ta ne pretrga. Slednje krivulje pripadajo ne-periodičnim rešitvam - kroženju.

Kot omenjeno je nihajni čas odvisen od začetne lege, kar sem preveril tako z analitično rešitvijo kot tudi z Verlet metodo in rezultate prikazal na sliki \ref{viju}

\begin{figure}[H]
\begin{center}
    \includegraphics[width=10cm]{graphs/t0.pdf}
    \caption{Nihajni čas v odvisnosti od začetnega kota}
    \label{viju}
\end{center}
\end{figure}

\subsection{Primerjava metod}

Ko sem spoznal rešitve našega problema, sem se lotil raziskovanja uspešnosti različnih metod. Ker sem imel še iz prejšnje naloge že veliko narejenega, se mi je zdela implementacija preveč trivialna, da bi bila zanimiva - tako sem le uporabil implementacije metod pripete v spletni učilnici.

Kvaliteta metode je seveda odvisna od tega, kakšno natančnost doseže v kakšnem času. Tako sem za različne metode narisal grafe maksimalnega odstopanja od analitične rešitve v prvih 15 sekundah v odvisnosti od časa, ter graf potrebnega časa v odvisnosti od dolžine koraka. Rezultati so vidni na sliki \ref{double}

\begin{figure}[H]
\begin{center}
    \includegraphics[width=0.8\linewidth]{graphs/double.pdf}
    \caption{Graf maksimalnega odstopanja od analitične rešitve v prvih 15 sekundah v odvisnosti od časa, ter graf potrebnega časa v odvisnosti od dolžine koraka}
    \label{double}
\end{center}
\end{figure}

Kot smo spoznali pri predavanjih, je glavni problem ne-simplektičnih metod ta, da se dlje časa kot simulacija traja - bolj se "kvarijo", zato sem si v naslednjem koraku pogledal, kako se za različne metode spreminja odstopanje amplitude od analitičnih predvidevanj. Za korak sem pri vseh metodah vzel $0.01$. Na ta način sem dobil graf \ref{triple}

\begin{figure}[H]
\begin{center}
    \includegraphics[width=12cm]{graphs/triple1.pdf}
    \caption{"Kvarjenje" amplitude skozi čas, oziroma skozi število nihajev. Lepo se vidi, da se pri obeh simplektičnih metodah amplituda čez čas "popravlja" medtem, ko se pri ostalih metodah čez čas le kvari. Pri eulerjevi metodi celo toliko, da po približno 40 nihajih ni bilo več mogoče razločiti, kje naj bi bili maksimumi (potrebno za odčitanje amplitude).}
    \label{triple}
\end{center}
\end{figure}

Kot omenjeno v uvodu pa so simplektične metode tako uspešne, ker v teoriji ohranjajo energijo sistema. To se seveda želel v naslednjem koraku preveriti, zato sem grafiral še absolutno razliko med energijo izračunano preko posamezne metode in zqčetno energijo. Na ta način sem dobil grafa na sliki \ref{triple2}, ki potrdi pričakovano. Zanimivo sicer opazimo majhno nihanje energije pri simplektičnih metodah, a pomembneje opazimo, da se odstopanje ustali pri neki natančnosti (ki je odvisna od izbire dolžine koraka), medtem, ko se energija pri ostalih metodah "kvari"\ skozi čas.

\begin{figure}[H]
\begin{center}
    \includegraphics[width=.8\linewidth]{graphs/triple.pdf}
    \caption{Absolutno odstopanje od zažetne energije sistema čez 3000 sekund (zgoraj) in samo čez 50 sekund (spodaj)}
    \label{triple2}
\end{center}
\end{figure}

Ko sem opazoval na kakšen način se energija dejansko kvari, sem odkril, da ne-simplektične metode delujejo, kot da bi imeli dušeno nihanje - amplitude se čez čas manjšajo, in prav tako energija sistema. Najbolj elegantno se mi je zdelo to ponazoriti z faznim portretom izračunanim s simplektično metodo (Verlet) in ne-simplektično (Runge-Kutta 4. reda). Medtem, ko simplektična ostane na originalni orbiti, ne-simplektična teži proti ravnovesni legi. Fazni portret je prikazan na sliki \ref{phase}, a efekt je še bolje viden na animaciji, ki je priložena poročilu.

\begin{figure}[ht]
\begin{center}
    \includegraphics[width=12cm]{graphs/single.pdf}
    \caption{Fazni portret pridobljen z simplektično metodo (Verlet) in ne-simplektično metodo (RKU4). Opazujemo, kako ne-simplektična metoda teži proti ravnovesni legi, medtem ko simplektična ohranja orbito (energijo)}
    \label{phase}
\end{center}
\end{figure}

\section{Zaključek}

Naučiti se o simplektičnih metodah se mi je zdelo zelo zanimivo, čeprav smo o njih slišali že pri Mafiji 1 in Numeričnih metodah. Tokrat je bilo zame prvič, da sem jih uporabil in izkusil njihovo uporabnost.

Kljub temu, pa me naloga ni najbolj navdihnila, saj se mi je zdela bolj kot samo nadeljevanje prejšnje naloge. Tu moram sicer priznati, da si zaradi različnih okoliščin nisem mogel vzeti časa za dodatno (in dodatno dodatno) nalogo, ki bi mi verjetno precej bolj popestrila življenje.

Vseeno se mi zdi, da sem nalogo opravil uspešno in vesel sem, da sem spoznal še eno orodje, ki mi zna priti prav v karieri fizika.



\end{document}